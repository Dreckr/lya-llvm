\documentclass[11pt, oneside, titlepage]{report}

  \usepackage{geometry}
  \geometry{letterpaper}
  \usepackage{amstext}
  \usepackage{amsmath}
  \usepackage{hyperref}

  \usepackage{graphicx}

  \usepackage{amssymb}

  \usepackage[utf8]{inputenc}
  \usepackage[portuges,brazilian]{babel}
  \usepackage[T1]{fontenc}
  \usepackage{titling}
  \usepackage{titlesec, blindtext, color}

  \renewcommand\maketitlehooka{\null\mbox{}\vfill}
  \renewcommand\maketitlehookd{\vfill\null}

  \newcommand{\hsp}{\hspace{20pt}}
  \titleformat{\chapter}[hang]{\Huge\bfseries}{\thechapter\hsp{|}\hsp}{0pt}{\Huge\bfseries}

  \makeatletter
  \let\Hy@linktoc\Hy@linktoc@none
  \makeatother

  \title{%
    MC032 - Estudo Dirigido \\
    \large \textit{Representação Intermediária LLVM} \\
    \small Guido Araujo
  }

  \author{
    Diego Rocha \\
     \texttt{RA 135494}
  }

  \begin{document}

    \begin{titlepage}

      \maketitle

    \end{titlepage}

    \tableofcontents

    \chapter{Introdução}

      O projeto LLVM foi iniciado em 2000 na Universidade de Illinois em Urbana-Champaign como um \textit{toolkit} de
      infrasestrutura para compiladores e oferece uma coleção de ferramentas modulares e reutilizáveis para a construção
      de compiladores \cite{llvm}. Ele oferece todas as camadas intermediárias de um compilador, bastando o
      desenvolvedor construir um \textit{front end} que transforma um programa em sua
      \textit{LLVM Intermediate Representation} \cite{llvmref}.

      O projeto ganhou relevância e teve seu desenvolvimento acelerado com a decisão da Apple Inc. em utilizá-lo como
      parte integral de suas ferramentas de desenvolvimento para os sistemas operacionais \textit{macOS} e \textit{iOS}.
      Atualmente o projeto possuí implementações de \textit{front ends} para diversas linguagens de programação e é
      amplamente utilizado na indústria por sua facilidade de uso, flexibilidade e qualidade de \textit{output}
      \cite{llvmfeatures}.

      Nas disciplinas MC910 (Construção de Compiladores) e MC911 (Projeto em Compiladores) da Universidade Estadual de
      Campinas são lecionadas técnicas para a construção de compiladores. Em especial, na disciplina MC911 é dado um
      projeto para a construção de um compilador para a linguagem de programação Lya \cite{lyagrammar} criada pelo
      Prof. Márcio M. Pereira.

      Para este estudo dirigido, foi proposto experimentar utilizar ferramentas LLVM na
      construção de um compilador para a linguagem Lya com o intuito de verificar sua viabilidade e reformular a
      disciplina MC911.

    \chapter{Objetivo}

      Este estudo dirigido teve como principal objetivo o aprendizado da representação intermediária LLVM e como ela
      pode ser utilizada com o auxílio das ferramentas LLVM na construção de compiladores para linguagens de
      programação.

      A representação intermediária LLVM é descrita por conjuntos de instruções da forma
      SSA (\textit{Static Single Assignment}) em blocos utilizados para a definição de funções e controle de fluxo.
      Estas características permitem que seja fácil transformar um programa em sua forma
      AST (\textit{Abstract Syntax Tree}) para a representação intermediária e que, então, otimizadores operem sobre a
      representação intermediária de forma eficiente e gerem códigos em linguagem de máquina performáticos.
      Além disso, esta representação é independente de plataforma, permitindo que compiladores que fazem uso do LLVM
      possuam múltiplas plataformas como \textit{target} com pouco esforço.

      Identificar a viabilidade destas ferramentas para a construção de um compilador para a linguagem Lya e sua
      facilidade de uso para o ensino permitirá que os alunos que cursarem MC911 aprendam a implementar compiladores
      de alta performance para qualquer linguagem de programação como é feito por grande parte da indústria, além de
      facilitar sua entrada em projetos de compiladores já consolidados.

    \chapter{Metodologia}

    \chapter{Compilador Lya}

      \section{Linguagem Kaleidoscope}

      \section{Análise Sintática}

        \subsection{\textit{Lexer}}

        \subsection{\textit{Parser}}

          \subsubsection{\textit{AST}}

      \section{Análise Semântica}

        \subsection{Visitors}

        \subsection{Enriquecimento da AST}

      \section{Geração de Código \small(\textit{LLVM IR})}

        \subsection{\textit{SSA Form}}

        \subsection{Variáveis Mutáveis}

        \subsection{Operações Aritméticas e Binárias}

        \subsection{Controle de Fluxo}

          \subsubsection{Condicionais}

          \subsubsection{Loops}

        \subsection{Procedimentos}

          \subsubsection{Definição de Procedimentos}

          \subsubsection{Chamadas de Procedimentos}

      \section{\textit{JIT compilation}}

        \subsection{Biblioteca da Linguagem}

    \chapter{Discussão}

      Ao fim deste estudo dirigido obtivemos um compilador funcional para um subconjunto da linguagem Lya utilizando as
      ferramentas do LLVM através de sua representação intermediária. Este compilador ainda não tem todas as
      funcionalidades que se espera ter um compilador completo, mas serve como uma boa prova de conceito sobre o que
      pode ser feito com o uso das ferramentas LLVM.

      A representação intermediária se mostrou bem fácil de ser gerada a partir da AST de um programa, seu conjunto de
      instruções é sucinto e de fácil entendimento e o projeto LLVM possuí boa documentação. Em contraste, o projeto
      llvmlite não está tão amadurecido e algumas funcionalidades da representação intermediária são difíceis de serem
      acessadas através dele como, por exemplo, a definição de constantes de \textit{string}.

      Com a criação de ferramentas auxiliares e uma boa documentação para iniciantes, acreditamos que a aplicação destes
      aprendizados em futuras turmas de MC911 só irá trazer benefícios aos alunos, deixando-os mais capacitados para
      trabalharem em projetos fora da disciplina e mais engajados com a disciplina por terem resultados mais rápidos.

    \phantomsection
    \addcontentsline{toc}{chapter}{Referências Bibliográficas}
    \bibliographystyle{unsrtnat}
    \begin{thebibliography}{9}

      \bibitem{llvm}
      LLVM Developers. LLVM Home Page.\\
      \texttt{http://llvm.org/}

      \bibitem{llvmref}
      LLVM Developers. LLVM Language Reference Manual.\\
      \texttt{http://llvm.org/docs/LangRef.html}

      \bibitem{llvmfeatures}
      LLVM Developers. LLVM Features.\\
      \texttt{http://llvm.org/Features.html}

      \bibitem{llvmtutorial}
      LLVM Developers. LLVM Tutorial.\\
      \texttt{http://llvm.org/docs/tutorial/index.html}

      \bibitem{lyagrammar}
      Márcio M. Pereira. Lya BNF Grammar.\\
      \texttt{https://iviarcio.wordpress.com/lya-bnf-grammar/}

      \bibitem{llvmlitedocs}
      Numba Developers. llvmlite documentation.\\
      \texttt{http://llvmlite.readthedocs.io/en/latest/index.html}

      \bibitem{pykaleidoscopegithub}
      Eli Bendersky. Pykaleidoscope GitHub Repository.\\
      \texttt{https://github.com/eliben/pykaleidoscope}

      \bibitem{llvmlitebasicusage}
      Eli Bendersky. Building and using llvmlite - a basic example.\\
      \texttt{https://eli.thegreenplace.net/2015/building-and-using-llvmlite-a-basic-example}

    \end{thebibliography}

  \end{document}